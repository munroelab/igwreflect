\documentclass[11pt,letterpaper]{article}
\usepackage[latin1]{inputenc}
\usepackage{amsmath}
\usepackage{amsfonts}
\usepackage{amssymb}
\usepackage{graphicx}

\title{Vertical Trapped Internal Waves in a Horizontal Channel}
\author{Prajvala K. Kurtakoti, Fan Lin, and James R. Munroe}

\begin{document}
	\maketitle
	
	\section{Introduction}
	
	Internal waves in a continously stratified fluid are often described in terms of vertically trapped.  This paper report on a series of laboratory experiments exploring the dynamics of such waves propagating and reflecting in a horizontal channel.
	
	 Are the waves generated truely mode-1?  
		- Need the vertical profile of density vs depth 
		- Need to solve the eigenvalue problem to calculate the vertical modes.
		- Need to measure internal wave displacement/velocity
		- Need to project internal wave onto different vertical modes
		- What fraction of energy is captured as 'mode-1'
		- How does this distribution change in time?
		- 20 experiments
		- vary stratification (6), vary frequency (8), rep (2)

	What is the dissipation along the tank as a function wave number?	
	   - measurements of internal waves at two horizontal position (1m apart)
	   - measure energy at a second location and take the ratio
	       - how does this vary with wave number/frequency
	       - expect $k^-3$ decay based on other paper... 
	   - does this energy go to heat or mixing?
	       - temperature change in tank? measure with probe
	            - estimate temp change expected if all energy went to heat.?
	       - measure stratification after experiment, compare differences
	   
	  How does the stratification change over time? hour to hour, day to day?
	   	 - what are the changes?
	   	 - heating, cooling? (room temp, humidity, pressure)
	   	 - evaporation?
	   	 - diffusion?
	   	 - I expect this not to be important over tens of minutes but important over several days
	   	 - (3 experiments, automatic profiles of density)
	   	 
	 What is the particle path of neutrally buoyany tracers? 
	    -- is the path circular? what is the theory?
	    -- Do particle tracers agree with synthetic schlieren measurements?
	    -- is there a second order effect
		   -- what is the theory for finite amplitude drift
		   -- What happens in finite sizes drifters? 
		       -- Initial phase of incoming wave    (4)
		       --- initial position of sphere, horizontal vs vertical (4)
		       --  size of sphere (3)
		        --  frequency of wave (6)
	   	
	 What happens to a sphere left in in the middle of the tank? 
	     Will it move, and  why? Is there a surface current?
	     -- while I expect nothing to happen,  we observed motion
	         -- rotation of earth
	         -- air currents
	         -- diffussion
	         -- surface effects around the sphere? (cf. with other work)
	   -- to discern cause need addtional experiments. (3)
	  
	In a vertically trapped wave what happens when it reflects off an end-wall
	  -- modal decomposition (shape)
	  -- how much energy reflected vs absorbation?
	     -- expect 100 at vertical wall and max absorbtion at critical wave angle.
	  -- need to separate incoming from outgoing waves (Hilbert transform)
	     -- how does modal energy distribution change
	    -- vary frequency (8), slope (5), stratification (3),  repetition (3)
	   - measure near end wall and 1-m away in same experiment
	   
	      -- when do internal waves reflect back off wave generator?
	           -- conditions for resonance?
	     
     A sloping boundary may cause internal flows. 
	 
	 How long does the system take to come to rest, when the wave maker is turned off?
	 
	 What happens when the end wall has a critical slope? Is absorbtion greater than reflection?
	 
	 Does mixing increase with increased slope?  
	

	
	\section{Methods}
	\section{Analysis}
	
\end{document}